\documentclass[fleqn,12pt]{article}
\usepackage{natbib}
\usepackage{amsthm,amsfonts,amsopn,amsmath,amssymb,vmargin}

\RequirePackage[pagebackref=true]{hyperref}
\RequirePackage{hypernat}
\hypersetup{colorlinks=true,linkcolor=blue,urlcolor=blue,citecolor=red,pdftitle=EIO,pdfauthor=Jaap Abbring Tobias Klein,pdfdisplaydoctitle=true}

\setpapersize{A4} \setmargrb{25mm}{18mm}{25mm}{23mm}
\renewcommand{\baselinestretch}{1.2}

\title{Female Labor Supply and Childcare Subsidies}
\author{An Example Proposal for the Final Assignment of EIO2 by JA \& TK}
\date{}

\begin{document}
\maketitle

We will study how the level of childcare support affects female labor supply.

To this end, we consider a finite horizon model of female labor supply in which each woman starts with neither kids nor work experience and makes annual decisions to work or not. Annual earnings depend on work experience following a Mincer-like equation. Kids arrive exogenously and increase the utility cost of working; childcare support limits this increase. Concretely, we take the model (but not the type of data!) in \citet[][Section 3]{ecma11:ecksteinwolpin} and allow childcare support to mitigate the effect of motherhood on the utility cost of working. To keep things manageable, we will simplify that model in many dimensions:
\begin{itemize}
\item We will make marriage dynamics irrelevant by setting  $M_t=1$ throughout.
\item We will simplify fertility by assuming that it only matters whether a woman has children or not and that, as long as she has none, they arrive with a probability that may depend only on her age. 
\item In the specification of earnings, we will ignore education ($S$) and earnings shocks ($\varepsilon_t=0$). Moreover, we will assume that a job providing these earnings is always available (that is, we will ignore the specification of a job offer probability).
\item In the specification of utility, we will only keep the main effects of consumption ($x_t$) and work ($(\alpha_1+v_t)p_t$), plus an effect of motherhood on the disutility of work mediated by childcare support. We will let $v_t$ have a logistic distribution.
\end{itemize}

We will simulate data on a panel of women from the time they leave school at age 22 to retirement at age 70. The panel data provide annual observations of employment and fertility. Moreover, for each woman, it indicates whether she is covered by a child support program (throughout her life) or not. Wages and earnings are not observed. 

\bibliographystyle{chicago}
\bibliography{examples}
\end{document}
