\documentclass[fleqn,12pt]{article}
\usepackage{natbib}
\usepackage{amsthm,amsfonts,amsopn,amsmath,amssymb,vmargin,verbatim,setspace}
\usepackage{pgf,pgfplots,tikz}
\usepackage{versions}

\RequirePackage[pagebackref=true]{hyperref}
\RequirePackage{hypernat}

%Choose whether to typeset figures in color or black and white.
\excludeversion{grayfig}
\includeversion{colorfig}

\pgfdeclareimage[height=7cm]{toPic}{2017/readyMixConcrete}
%\pgfdeclareimage[height=5.9cm]{cinema}{2016/tuschinski}
%\pgfdeclareimage[height=5cm]{rvtwo}{2015/rvDealer2}


\usetikzlibrary{arrows,calc,fit,matrix,positioning,shapes.multipart,shapes.symbols}
\hypersetup{colorlinks=true,linkcolor=blue,urlcolor=blue,citecolor=red,
    pdftitle=EIO,
    pdfauthor=Jaap Abbring Tobias Klein,
    pdfsubject=JEL Codes C14 C41,
    pdfkeywords=,
    pdfdisplaydoctitle=true}

%\setpapersize{USletter}\setmargrb{25mm}{18mm}{25mm}{23mm}
\setpapersize{A4} \setmargrb{25mm}{18mm}{25mm}{23mm}
%\doublespace
\renewcommand{\baselinestretch}{1.2}

\let\oldthebibliography=\thebibliography
\let\endoldthebibliography=\endthebibliography
\renewenvironment{thebibliography}[1]{\begin{oldthebibliography}{#1}\setlength{\itemsep}{4.2pt}}{\end{oldthebibliography}}

\newcommand{\pos}{tbp}

\newcommand{\E}{\mathbb{E}}
\newcommand{\sign}{\mathrm{sgn}}
\newcommand{\R}{\mathbb{R}}
\newcommand{\Rp}{[0,\infty)}
\newcommand{\Rpp}{(0,\infty)}
\newcommand{\Rpext}{[0,\infty]}
\newcommand{\Rppext}{(0,\infty]}
\newcommand{\C}{\mathbb{C}}
\newcommand{\N}{\mathbb{N}}
\newcommand\setmapsto{\rightarrow}
\renewcommand\L{{\cal L}}
\newcommand\Linv{{\cal K}}

\newcommand{\var}{\mathrm{var}}

\def\CC{C\texttt{++}\ }
 
\theoremstyle{plain}
\newtheorem{theorem}{Theorem}
\newtheorem{lemma}{Lemma}
\theoremstyle{definition}
\newtheorem{definition}{Definition}

   \newcounter{keepenumi}
%\renewcommand{\theenumi}{(\roman{enumi})}
\newcommand{\cites}[1]{\citeauthor{#1}'s (\citeyear{#1})}


\begin{document}
\section*{Teacher incentives}

\citet{aer12:dufloetal} used experimental data to study the effects of monitoring and incentive pay on teacher absence and learning in India. In the experiment's treatment group, daily attendance is closely monitored and teachers earn $\pi(l)=500+50\cdot\max\{0,l-10\}$ rupees at the start of the month for $l$ days of work in the preceding month. \citeauthor{aer12:dufloetal} first studied the effects of monitoring and incentive pay using simple experimental comparisons between the treatment and control groups (the latter did not get the same level of monitoring nor incentive pay). Then, they estimated a daily labor supply model using data on the treatment group, which allows to extrapolate the experimental findings to other wage schedules than the experimental one. 

Your task is to develop and analyze a simplified and slightly adapted version of this model. In particular, focus on a teacher's work decisions in a single month. On each of the month's 26 work days $t=1,\ldots,26$; the teacher decides to go to school ($d_t=1$) or not ($d_t=0$) based on the number of days $l_t$ already worked, the number of days $z_t$ remaining in the month, and a vector of independent and mean-zero type 1 extreme value utility shocks $\varepsilon_t\equiv(\varepsilon_{0,t},\varepsilon_{1,t})$; earning utility $(1-d_t)\mu+\varepsilon_{d_t,t}$. On the model's final day, day 27, the teachers earns $\pi(l_{27})$, which gives utility $\gamma\pi(l_{27})$.  The teacher makes choices that maximize the expected sum of utilities, without discounting (this and the linearity of utility justify that you ignore the nonwork days between the 26 work days of the model). The result is a version of the basic model without serial correlation in \citet[][Section III]{aer12:dufloetal} with linear utility, extreme value errors, and a zero risk of being fired. Note that we used slightly different notation.

In your analysis and report you may want to evaluate one or more other wage schedules than the experimental one. You may also want to reflect on the value of using a dynamic model over  using a static model in which decisions are simply based on the marginal effect of working on pay on day 27. 

\bibliographystyle{chicago}
\bibliography{final}
\end{document}
